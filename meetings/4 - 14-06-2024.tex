\documentclass[11pt,a4paper]{article} 

\usepackage[english]{babel} %needs to specified for minutes package (else it will be in German)
\usepackage{a4wide}%For a wider spacing of the text (smaller left/right margin)
\usepackage{amssymb}% http://ctan.org/pkg/amssymb  for the check and cross symbol
\usepackage{pifont}% http://ctan.org/pkg/pifont    for the check and cross symbol
\usepackage{setspace}
\usepackage{minutes}

\pagestyle{plain}

%more memorable commands to make the checked and crossed symbols
\newcommand{\done}{\ding{51}}%
\newcommand{\fail}{\ding{55}}%

\begin{document}

\section{Agenda}
\begin{itemize}

\item Meeting 14 juni 2024 09:00 – 10:00, Plaats: C4.112

\item Voorzitter: RvB\footnote{RvB: Roan van Brussel}, Notulen: AJ\footnote{AJ: Arie Jongejan}
\item Present:
\begin{itemize} 
\item Prof. Lex Kaper, Stephanie Heikamp, Omar Ould Boukattine
\item RvB, AJ, TJ\footnote{TJ: Tommy Jones}
\end{itemize}
\end{itemize}
Sterrenkundeproject:  Groep 02-Open Clusters M44 

\begin{enumerate}
    \item Opening
    \item Mededelingen
    \item Vaststellen agenda
    \item Goedkeuren notulen 06/06/24 (bijgevoegd)
    \item Oude actiepunten (zie To-do lijst, bijgevoegd)
    \item Voortgang
    \begin{enumerate}
        \item Sterren binnen afbeeldingen gevonden. De members gematched met de gegeven databases. (TJ)
        \item Magnitudes van de gevonden sterren bepaald, problemen met ijking lijken opgelost. Presentatie colour-colour diagram zowel voor hele dbase als voor door ons gematchte M44 sterren. (RvB)
        \item Bepaling interstellaire extinctiefactor even on-hold gezet.
        \item Animatie/poster, provisioneel story board gemaakt
        
    \end{enumerate}
    \item Onderzoeksvraag
    \begin{enumerate}
    \item Kunnen we blue stragglers identificeren in het colour-magnitude diagram?
    \item en zo ja, kunnen we dan nog iets betekenisvols zeggen over hun plaats en/of omgeving?
    \end{enumerate}
     \item Planning
    \item WVTTK
    \item actiepunten
    \item volgende afspraak/meeting
    
\end{enumerate}

\pagebreak
\begin{itemize}
    \item 
 2 Notulen vergadering 06/06/24 


2.1 Waar we nu zijn:
 We hebben observaties en stacks met i’, r’ en g’ gemaakt waarbij we de sterren kunnen
 detecteren en vergelijken met een database aan ”members” van M44. Dit komt uit een
 onderzoek van Jeison Alfonso en Alejandro Garc´ ıa-Varela in 2023. Daaruit willen we kijken
 naar (blauwe) stragglers in een HR-diagram.

 
 2.2 Matchen objecten uit observaties aan database en magnitude vinden:
 We kunnen bij het koppelen van sterdata gebruik maken van de gegeven parallax uit de Gaia
 database van een of meerdere sterren en kunnen dan een zero point correction uitvoeren om
 de magnitude van alle sterren te berekenen. Dit moet wel in verschillende filters. Er zijn ook
 sterren die in realiteit deel zijn van de cluster die niet opgenomen zijn in Gaia door een te
 hoge magnitude of andere factoren. Het is ook interessant om te zien wat de zwakste ster is
 die we hebben kunnen detecteren.

 
 2.3 Extinctie:
 Interacties tussen Gas/Stof en licht tussen ons en de sterren zal de Hertzprung-Russel (HR)
 diagram aanpassen van de realiteit. We moeten dus onze data calibreren met de extinctie.
 Dit is voor elke filter anders. We kunnen hiervoor ”kleur-kleur diagrammen” maken. Hiermee
 kunnen we de extinctie vinden. Een aantal opties hiervoor zijn g’-r’ of r’-i’.

 
 2.4 Stragglers:
 (Blue) Stragglers blijken anders te evolueren dan andere sterren. Lex is van mening dat dit
 komt door botsingen maar literatuur stelt dat er verschillende theorie¨en zijn voor dit gedrag.
 De primaire theorie stelt dat botsingen tussen sterren zorgen voor een ”verjongingskuur” van
 de ster wat leidt tot ongebruikelijke sterevolutie, wat zorgt tot het gedrag dat we zien van
 deze stragglers. Deze botsingen komen waarschijnlijk uit dubbelstersystemen die de norm
 blijken te zijn bij zware sterren. In het algemeen wordt de evolutie van clusters niet goed
 begrepen en is een actief onderzoeksveld.

 
 2.5 Isochronen:
 Isochronen zijn een manier om meer informatie uit de cluster te krijgen. Hiermee kunnen
 we de leeftijd van een cluster vinden. Er is een bepaalde ”turn-off” punt waar de sterren
 afbuigen van hoofdas staan waaruit de leeftijd afgeleid kan worden met een fit. Stragglers
 passen daar niet goed bij door hun verjongingskuur. Parsec modellen worden gebruikt om een
 f
 it uit te voeren op de data om de leeftijd te vinden van een cluster. Deze zijn te genereren
 in http://stev.oapd.inaf.it/cgi-bin/cmd. We hebben hierbij wel de extinctie-waardes
 nodig

 \item 3 Antwoorden op specifieke vragen (deel 4 van agenda)

 
 1. Lijkt dit een zinvolle onderzoeksrichting?
 Onderzoek in de evolutie van open clusters is een actueel onderzoeksgebied en vooral
 het bestaan van stragglers is een grote vraag in de sterrenkunde.

 
 2. Zijn er voldoende meetbare objecten (stragglers) binnen onze data set?
 Er zijn een redelijk aantal stragglers in een cluster en deze zijn vaak ook helderder dan
 andere sterren doordat ze meer massa hebben door de botsingen. De enige vraag is of
 we deze sterren binnen ons zichtveld hebben opgenomen doordat we niet de hele cluster
 hebben kunnen waarnemen.

 
 3. Hoe identificeren we die van de rest (2.c)? (N.B. We hebben de membership lijst van
 de hele cluster!)
 Stragglers zijn te zien in HR-diagrammen en Isochronen als datapunten die op locaties
 zitten waar ze niet horen te zitten. Dit is waarom ze ”stragglers” genoemd worden. Ze
 dwalen af van het normale pad van een ster door de botsingen.

 
 4. Zijn er andere eigenschappen (anders dan temperatuur/kleur) die blue stragglers on
derscheiden van andere objecten binnen de cluster (snelheid, plaats in 3D of in hyper
sphere,. ..)?
 Hier hadden we niet echt een direct antwoord op gekregen, maar dit kunnen we bij de
 volgende meeting opnieuw opbrengen als vraag.



 
 \item 4 TODOlijst
 Dit is best grof maar geeft ons een grof idee wat er gedaan moet worden.

 
 1. Sterren vinden/detecteren uit stacked data

 
 2. Gevonden sterren vergelijken met database om members van M44 te vinden

 
 3. Magnitude van sterren vinden in verschillende filters

 
 4. Extinctie van sterrenstof en atmosfeer (airmass) vinden bij verschillende filters

 
 5. Isochroon opstellen en fitten in g’-r’ of r’-i’

 
 6. HR diagram maken

 
 7. Blue stragglers vinden

 
 8. Meer analyse doen op de gevonden Blue stragglers (hier komen we later nog op terug)
\end{itemize}

\end{document}